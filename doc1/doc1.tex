\documentclass{article}
% il comando inputenc serve ad interpretare correttamente i caratteri immessi nell'editor
% utf-8 è la codifica di input
\usepackage[utf8]{inputenc}
% impostare la codifica dei caratteri
% poiché latex è stato inventato per scrivere in inglese bisogna importare i caratteri di codifica occidentali
%per fare ciò bisogna importare (usare) un package
\usepackage[T1]{fontenc}
% il comando precedente usa la codifica t1
% e dice al compilatore mediante fontenc (font encoding) di codificare i caratteri in tale codifica
% per scrivere ò senza questa codifica si sarebbe dovuto scrivere \'o

%l'ultimo pacchetto da utilizzare è quello della codifica delle parole e del riconoscimento delle lingue = babel
%importiamo la linguea italiana
\usepackage[italian]{babel}

\title{Documento 1}
\author{Antonio Maulucci http://www.antomau.com}
\date{February 2017}

\begin{document}

\maketitle

% questo è un commento
% (C) Antonio Maulucci 2017
% http://www.antomau.com
% il carattere '\' serve ad inizializzare un comando

%l'estensione dei documenti latex è .tex


\section{Introduction}
\section{Sapessi}
\section{califragilisti}

The logo of Latex is: \LaTeX


%vediamo coem scrivere solo un piccolo pezzo di testo in lingua straniera
\foreignlanguage{english}{Hello!}

%per porizioni più lunghe di codice in altra lingua
\begin{otherlanguage*}{english}
Hello, my name is Antony and I'm writing using \LaTeX language!
\end{otherlanguage*}

\end{document}
