%
% la struttura del file sorgente
%
% il file sorgente di Latex deve essere strutturato in maniera precisa
%
% primo elemento fondamentale
% dichiarazione di classe
% documentclas....
%
%dichiarazione di inizio e fine documento
% begin e end
%
% le istruzioni tra documentclass e begin
% sono il preambolo del documento
% vengono caricati i pacchetti
% vengono definiti i comandi e gli ambienti personalizzati
% si definiscono le opzioni generali del documento
%

\documentclass[a4paper]{article}
\usepackage[utf8]{inputenc}
\usepackage[T1]{fontenc}
\usepackage[italian]{babel}

\title{doc3 \dots} % il comando dots mette 3 puntini di sospensione
\author{Antonio Maulucci \and Antonio Maulucci} % il comando and divide i due nomi con uno spazio predefinito
\date{February 2017}

% inizio del documento
\begin{document}

%
% il comando maketitle produce la stampa del titolo e degli autori e della data
%
\maketitle

% il comando section produce un titolo di sezione
\section{Il file sorgente di \LaTeX}

\section{Spazi e righe vuote}

Il carattere tabulatre viene visto come uno spazio

Più spazi consecutivi costituiscono un solo spazio

Una sola andata a capo è considerata come uno spazio

Più righe vuote vengono trattate come una sola riga vuota

\section{I commenti}

I commenti su una riga possono essere inseriti mediante il carattere $\%$

Mentre per inserire i commenti su più righe si utilizza un apposito ambiente denominato "comment"

\subsection{Sottoparagrafo}

Questo è un sottoparagrafo

\begin{comment}

questo
è
un
commento
su
più righe

\end{comment}

% fine del documento
\end{document}

dopo la fine del documento
il compilatore latex esclude tutto il contenuto e quindi si può scrivere ciò che si vuole
... appunti e altro....
