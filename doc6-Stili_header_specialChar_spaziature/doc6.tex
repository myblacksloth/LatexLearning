\documentclass{article}
\usepackage[utf8]{inputenc}
\usepackage[T1]{fontenc}
\usepackage[italian]{babel}

\title{doc6}
\author{Antonio Maulucci \and Antonio Maulucci}
\date{\dots February 2017}

\begin{document}

\maketitle



\section{Materiale iniziale \dots pagina 30}

\backslash frontmatter = non numera le sezioni ma

numera le pagine con i numeri romani minuscoli

\backslash mainmatter = numera pagine e sezioni con numeri romani

\backslash backmatter - materiale finale: numera le pagine continuando

la numerazione del materiale principale

\section{Stili di pagina}

\backslash pagestyle\{style\}

PLAIN = numeri di pagina nel piede / testata vuota

EMPTY = piedi e testa vuoti

HEADINGS = pagine 31 / 32

\section{Gli Headeras}

\Huge{Huge}

\huge{huge}

\LARGE{LARGE}

\Large{Large}

\large{large}

\normalsize{normalsize}

\small{small}

\footnotesize{footnotesize}

\scriptsize{scriptsize}

\tiny{tiny}

\normalsize % altrimenti tutto il testo continua ad essere scritto in tiny

\section{Altri caratteri speciali}

\P

\copyright Antonio Maulucci

\pounds 127

\S

\O

\o

?'

%
% INSERIAMO UN INTERRUZIONE DI PAGINA

\newpage

\section{Interruzioni e spaziature}

\subsection{Interruzioni}

\subsubsection{newline}

Questa è una linea in cui vogliamo inserire un interruzione \newline per far andare il testo a caporigo

Questa è la medesima riga in cui vogliamo andare a caporigo \\ utilizzando il carattere di abbreviazione di \backslash newline

\subsection{Spaziature}

\subsubsection{Spaziatura verticale}

Mentre scrivo il testo voglio inserire una spaziatura verticare di 2cm \vspace{2cm} per continuare il testo più in basso

\subsubsection{Spaziatura orizzontale}

Mentre scrivo il testo voglio \\ ottenere una spaziatura orizzontale \\ di 2 cm \hspace{2cm} per continuare il testo più a destra

\vspace{72pt} % 1 punto tipografico equivale ad 1/72 pollice

Vedi pagina 19 del secondo manuale (impara \LaTeX)


\end{document}
