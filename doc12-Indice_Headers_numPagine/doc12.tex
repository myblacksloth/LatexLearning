\documentclass[a4paper, 12pt]{article}
\usepackage[T1]{fontenc}
\usepackage[utf8]{inputenc}
\usepackage[italian]{babel}
\usepackage{geometry}
\geometry{a4paper,top=2cm,bottom=2cm,left=1.5cm,right=1cm,bindingoffset=5mm}

\usepackage[colorlinks]{hyperref} % per ottenere i collegamente interni ed esterni alla pagina colorati

\usepackage{color} % per usare i colori


\usepackage[dvipsnames]{xcolor} % per usare i colori secondari



\hypersetup{colorlinks=true, linkcolor=darkgray, urlcolor=RedOrange} % per impostare il colore dei link
% linkcolor è riferito ai collegamenti interni alla pagina
% urlcolor è riferito ai collegamenti esterni e alle mail
%
% consultare questa pagina per comprenderne il funzionamento
% https://en.wikibooks.org/wiki/LaTeX/Hyperlinks#Customization
%



\usepackage{graphicx} % per utilizzare le immagini


\usepackage{tgpagella} %cambiare font all'intero documento



\usepackage{lastpage} %per poter ottenere il numero dell'ultima pagina del documento

\usepackage{fancyhdr} % per utilizzare gli header personalizzati
\pagestyle{fancy}
\rhead{Made with \LaTeX\ } %header pesonalizzato a destra
\lhead{made by Antonio Maulucci} % header personalizzato a sinistra
\rfoot{This is the page \thepage\ of \pageref{LastPage}} %footer personalizzato a destra
% il comando \thepage stampa il numero della pagina corrente
\lfoot{Feb 2017} %footer sinistro




\pagenumbering{roman} %numerazione romana delle pagine



\title{{\color{red} doc12}}
\author{Antonio Maulucci\\ \href{http://www.antomau.com}{antomau.com}}
\date{February 2017}

\begin{document}

\maketitle

\tableofcontents % questo comando crea l'indice dei contenuti del documento

\newpage

\section{Indice del documento}

Per creare un indice per il documento occorre utilizzare il comando

\begin{verbatim}
    \tableofcontents
\end{verbatim}

\section{Approfondimento sui colori}

\subsection{I colori primari}

I colori primari si inseriscono, come visto precedentemente, importando il package
\begin{verbatim}
color
\end{verbatim}

I colori primari sono:

\begin{itemize}
    \item black
    \item blue
    \item brown
    \item cyan
    \item darkgray
    \item gray
    \item green
    \item lightgray
    \item lime
    \item magenta
    \item olive
    \item orange
    \item pink
    \item purple
    \item red
    \item teal
    \item violet
    \item white
    \item yellow
\end{itemize}

\subsection{I colori secondari}

Per poter utilizzare i colori secondari occorre importare un package diverso, ovvero:

\begin{verbatim}
    \usepackage[dvipsnames]{xcolor}
\end{verbatim}

che comprende colori diversi da quelli visti precedentemente...\\ Per maggiori informazioni consultare \href{https://en.wikibooks.org/wiki/LaTeX/Colors}{questa guida}

\subsubsection{Esempi di colori secondari}

\begin{itemize}
    \item {\color{SeaGreen}SeaGreen}
    \item {\color{OrangeRed}OrangeRed}
    \item {\color{Goldenrod}Goldenrod}
    \item {\color{RedOrange}RedOrange}
\end{itemize}



\section{Headers \& Footers}

\subsection{Headers e Footers personalizzati}
Gli \textit{headers} sono le intestazioni in testa alla pagina; i \textit{footers} sono le note a piè di pagina.

\subsubsection{Il package}
Per usarle occorre importare il package
\begin{verbatim}
    fancyhdr
\end{verbatim}
proprio come è stato fatto in questo documento.\\Tale package impone che il contenuto della pagina dai suddetti elementi venga separato mediante l'utilizzo di una linea.

\subsubsection{Lo stile della pagina}
Poi occorre impostare lo stile di pagina
\begin{verbatim}
    \pagestyle{fancy}
\end{verbatim}

\subsubsection{I margini della pagina}
Poi occorre modificare adeguatamente la dimensione dei margini affiché il testo possa antrare nel documento. Ad esempio:
\begin{verbatim}
    \geometry{a4paper,top=2cm,bottom=2cm,left=1.5cm,right=1cm,bindingoffset=5mm}
\end{verbatim}

\subsubsection{Impostare gli header}

Header destro:
\begin{verbatim}
    \rhead{Made with \LaTeX\ }
\end{verbatim}

Header sinistro:
\begin{verbatim}
    \lhead{made by Antonio Maulucci}
\end{verbatim}

\subsubsection{Impostare i footers}

Footer destro:
\begin{verbatim}
    \rfoot{This is the page \thepage }
\end{verbatim}

Footer sinistro:
\begin{verbatim}
    \lfoot{Feb 2017}
\end{verbatim}






\subsection{Gli stili predefiniti}

Utilizzando lo stile di pagina 
\begin{verbatim}
    \pagestyle{headings}
\end{verbatim}
le intestazioni vengono inserite in maniera automatica.

\subsection{Altri stili}

Per altri stili consultare la guida \url{https://it.sharelatex.com/learn/Headers_and_footers}



\vspace{1cm}

\section{La numerazione delle pagine}

È possibile modificare lo stile della numerazione delle pagine.

\subsection{Indice alfanumerico}

Per impostare un indice alfanumerico nella numerazione delle pagine utilizzare il comando
\begin{verbatim}
    \pagenumbering{alph}
\end{verbatim}

\subsection{Indice in numeri Romani minuscoli}

Utilizzare il comando
\begin{verbatim}
    \pagenumbering{roman}
\end{verbatim}

\subsection{Impostare manualmente il numero di pagina}

utilizzare il comando seguente sostituendo il 3 con il valore desiderato
\begin{verbatim}
    \setcounter{page}{3}
\end{verbatim}

Nota Bene: La numerazione delle pagine seguenti continuerà dal numero\\ assegnato a questa pagina
\\\\
Nota Bene: Come si può notare è stata saltata la pagina 4 perché è stato impostato\\ manualmente iaffiché questa pagina corrisponda alla numero 5 del documento\\Per fare ioò è stato utilizzato il comando
\begin{verbatim}
    \setcounter{page}{5}
\end{verbatim}

\setcounter{page}{5} %imposto la numerazione di questa pagina in numero 4



\subsection{Personalizzare l'indice di pagina}

È possibile personalizzare l'indice di pagina sfruttando le peculiarità di \LaTeX\\ mediante alcuni comandi e mediante i footers
\\\\
Ad esempio per scrivere pagina x di x nel footer destro della pagina si può utilizzare il comando
\begin{verbatim}
    \rfoot{Page \thepage \hspace{1pt} of \pageref{LastPage}}
\end{verbatim}

dopo aver importato il package
\begin{verbatim}
    \usepackage{lastpage}
\end{verbatim}

ad esempio questo documnto sta usando il footer
\begin{verbatim}
    \rfoot{This is the page \thepage\ of \pageref{LastPage}}
\end{verbatim}

\subsection{Ulteriori annotazioni}

Per ulteriori annotazioni consultare \href{https://it.sharelatex.com/learn/Headers_and_footers}{questa guida}.

\subsection{Rimuovere il numero di pagina}

Per rimuovere il numero di pagina utilizzare il comando visto precedentemente\\ per impostare manualmente il numero di pagina e settarlo a 0
\begin{verbatim}
    \setcounter{page}{0}
\end{verbatim}


\vspace{2cm}

\section{Rimuovere lo stile di una pagina}

Per rimuovere lo stile della pagina corrente, compreso il numero di pagina utilizzare:
\begin{verbatim}
    \thispagestyle{empty}
\end{verbatim}





\vspace{4cm}
\begin{center}
{\fontfamily{qag}\selectfont
\copyright \sffamily{Antonio Maulucci 2017}
}
\end{center}
\end{document}
