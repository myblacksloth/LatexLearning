\documentclass{article}
\usepackage[T1]{fontenc}
\usepackage[utf8]{inputenc}
\usepackage[italian]{babel}

\title{doc8}
\author{Antonio Maulucci}
\date{February 2017}

\begin{document}

\maketitle

\section{I riferimenti}
% quando si inserisce una sezione si può anche inserire una label (etichetta) che la identifica
\label{etichtta} %label non di riferimento
\label{sez:refEtichetta} % label di riferimento

Nella prossima sezione vedremo come si usano i riferimenti

\section{Utilizzare i riferimenti}

Voglio fare riferimento alla sezione 1 quindi uso \backslash ref\{etichetta\} ??

ASSOLUTAMENTE NO PERCHEÉ LABEL NON E' UNA LABEL DI RIFERIMENTO

ALLORA DEVO PRIMA RICHIARARE UNA LABEL DI RIFERIMENTO PER POTER UTILIZZARE QUESTA FUNZIONE

Adesso posso fare riferimento alla sezione 1 in questo modo: riferimento a \ref{sez:refEtichetta}

% adesso intendo stampare la pagina in cui si trova il riferimento
che si trova nella pagina \pageref{sez:refEtichetta}

\newpage

\section{Gli ambienti di \LaTeX}

\subsection{ambiente em}

Questo ambiente mette il testo in evidenza

\begin{em}
Ecco un esempio!
\end{em}

\subsection{ambiente quotation}

Inserisce un rientro all'inizio di ogni paragrafo e dona al testo una spaziatura normale

\begin{quotation}
esempio riga 1!

esempio riga 2... questo testo viene scritto affinché occupi due righe per dimosrare come viene stampato
\end{quotation}

\subsection{ambiente quote}

Non inserisce un rientro all'inizio di ogni paragrafo e ha una spaziatura leggermente maggire

\begin{quote}
esempio riga 1!

esempio riga 2... questo testo viene scritto affinché occupi due righe per dimosrare come viene stampato
\end{quote}

\subsection{ambiente verse}

Usa una spaziatura alla pari dell'ambiente
\begin{em}
quote
\end{em}
ma le linee per andare a capo devono terminare con $\backslash \backslash$ tranne l'ultima riga di ogni strofa.

Le strofe sono separate da linee bianche

\begin{verse}
Cantami o diva, \\
del Pelide Achille \\
l'ira funesta.

Strofa nuvova.
\end{verse}

\subsection{Ambiente center}

Serve a centrare il testo in modo automatico

\begin{center}
Cantami o diva,

del pelide Achille....

Questo è un paragrafo nettamente centrato all'interno della corrente pagina scritta con \LaTeX
\end{center}

\newpage

\subsection{ambiente FlushLeft}

Questo ambiente effettua automaticamente l'impaginazione

\begin{flushleft}
paragrafo giustificato a sinistra in cui l'impaginazione avviene automaticamente come anche l'andata a caporigo
\end{flushleft}

\subsection{ambiente flushright}

\begin{flushright}
sto provando adesso l'ambiente flushright del linguaggio \LaTeX per vedere come viene impaginato il testo
\end{flushright}

\subsection{ambiente verbatim}

\begin{verbatim}
Questo ambiente non fa nulla e si comporta come \texttt

non interpreta neanche i caratteri speciali come \ che può essere scritto

tranquillamente senza utilizzare il comando \backslash

tutti i comandi in questa sezione sono nulli --> ad esempio \newline \newpage

tuttavia bisogna andare a capo manualmente altrimenti accade questo @@@@@@@@@@@

ovvero si esce dai margini della pagina
\end{verbatim}

\newpage










\section{Gli elenchi}

\subsection{Lista semplice: elenco puntato}

\begin{itemize}
\item primo elemento
\item secondo elemento
\item terzo elemento
\end{itemize}

\subsection{Elenco numerato}

\begin{enumerate}
    \item primo elemento
    \item secondo elemento
    \item terzo elemento
\end{enumerate}

\subsection{Elenco di descrizioni}

\begin{description}
    \item[itemize] crea un elenco puntato
    \item[enumerate] crea un elenco numerato
    \item[descriprion] permette di descrivere degli elementi
\end{description}

\subsection{Le tabulazioni}

Consultare il manuale 2 a pagina 31














\newpage

\section{Le Tabelle}

Quello delle tabelle è l'ambiente più potente di \LaTeX\ e si possono realizzare tutti i tipi di tabella

\subsection{allineamento delle colonne}

L : left\\
C : center\\
R : tight\\

\subsection{Tabella semplice (lcr)}

\begin{tabular}{lcr}
r1:c1 & r1:c2 & r1:c3 \\
r2:c1 & r2:c2 & r2:c3\\
r3:c1 & r3:c2 & r3:c3
\end{tabular}

\subsection{Tabella con il bordo (l|c|r)}

\begin{tabular}{l|c|r}
r1:c1 & r1:c2 & r1:c3\\
r2:c1 & r2:c2 & r2:c3
\end{tabular}

\subsection{Tabella con il bordo anche ai margini (|l|c|r|)}

\begin{tabular}{|l|c|r|}
r1:c1 & r1:c2 & r1:c3\\
r2:c1 & r2:c2 & r2:c3
\end{tabular}

\subsection{Tabella con il bordo per ogni casella (|l|c|r|)}
% | indica riga verticale
% \hline indica riga orizzontale
% & indica colonna successiva
% \\ indica riga successiva
\begin{tabular}{|l|c|r|}
\hline
r1:c1 & r1:c2 & r1:c3\\
\hline
r2:c1 & r2:c2 & r2:c3\\
\hline
r3:c1 & r3:c2 & r3:c3\\ %ometeendo l'andata a capo non uscirà la riga inferiore della tabella
\hline
\end{tabular}

\subsection{Una tabella particolare}
% questa è una tabella particolare
%
%il bordo sinistro ha una doppia rigatura
% p{Xcm} indica che tale colonna deve essere larga X centimetri
%doppia colonna al centro
% *{2}{c|}|} indica che che la dichiarazione "c|" deve essere seguita per 2 volte
% ovvero c| sarà applicata alle due colonne successive
% inserendo | si sommerà a quella della precedente colonna e quindi vi saranno due linee verticali
\begin{tabular}{||p{5cm}||*{2}{c|}|{r}||}
\hline
&  quantità & prezzo & disponibilità\\ %colonna di sinistra vuota per la prima riga
\hline
\bfseries smN9005 & 3 & 550 & NO\\
\hline
\bfseries smG935f & 1 & 829 & OK\\
\hline
\end{tabular}

\subsection{Table e Figure}

pagina 35 guida \LaTeX\ (imparare Latex)






\begin{center}
\copyright Antonio Maulucci 2017
\end{center}
\end{document}
