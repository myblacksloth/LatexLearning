\documentclass[a4]{article}
\usepackage[utf8]{inputenc}
\usepackage[T1]{fontenc}
\usepackage[italian]{babel}

\title{doc2}
\author{Antonio Maulucci}
\date{February 2017}

\begin{document}

\maketitle

\section{I caratteri speciali}

I seguenti caratteri sono ritenuti caratteri speciali, essi hanno bisogno di essere terminati mediante l'utilizzo di uno spazio o della barra rovescia

\LaTeX\ Logo latex

\TeX\ logo tex

\LaTeX  logo latex seguito da due spazi

%
% COMANDI PER LE SEZIONI DI TESTO
% I comandi del tipo \XX{argomento}
% formattano solo il testo tra parentesi graffe
% seguendo le istruzioni dichiarate
% es.: il testo "argomento" viene trattato usando le istruzioni XX
%
\section{Dichiarazione di una sezione}
Questa è una sezione di testo di "Dichiarazione..."

%
% LE DICHIARAZIONI
% Una dichiarazione inizia con il carattere '\'
% es.: \itshape
% una dichiarazione delimita un testo che deve seguire le regole specificate
% una dichiarazione si annulla solamente con un'altra dichiarazione
%

%
% I CARATTERI SPECIALI DI LATEX
%
\section{I caratteri speciali}

Il carattere back slash comincia un comando

I caratteri parentesi graffe delimitano un gruppo

Il carattere dollaro delimita una formula matematica in linea

Il carattere esponente delimita un esponente matematico

Vi sono anche altri caratteri... cosultare la guida

Il back slash si realizza con \textbackslash

e la parentesi graffa con $\{$

\section{Gli ambienti}

Un ambiente è una porzione di codice delimitata da un comando di apertura e uno di chiusura

\textbackslash begin $\{$ ambiente $\} [ ] \{ \}$

...

\textbackslash end

dove "ambiente" è il nome dell'ambiente

Gli ambienti possono essere annidati

\textbackslash begin$\{$ ambiente esterno $\}$

\textbackslash begin$\{$ ambiente interno $\}$

\textbackslash end $\{$ ambiente interno $\}$

\textbackslash end $\{$ ambiente esterno $\}$



\end{document}
