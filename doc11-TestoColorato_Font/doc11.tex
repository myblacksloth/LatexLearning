\documentclass[a4paper, 11p]{article}
\usepackage[T1]{fontenc}
\usepackage[utf8]{inputenc}
\usepackage[italian]{babel}
\usepackage{geometry}
\geometry{a4paper,top=1cm,bottom=2cm,left=1.5cm,right=1cm,bindingoffset=5mm}

\usepackage[colorlinks]{hyperref} % per ottenere i collegamente interni ed esterni alla pagina colorati

\usepackage{color} % per usare i colori

\hypersetup{colorlinks=true, linkcolor=blue, urlcolor=cyan} % per impostare il colore dei link
% linkcolor è riferito ai collegamenti interni alla pagina
% urlcolor è riferito ai collegamenti esterni e alle mail
%
% consultare questa pagina per comprenderne il funzionamento
% https://en.wikibooks.org/wiki/LaTeX/Hyperlinks#Customization
%



\usepackage{graphicx} % per utilizzare le immagini




\usepackage{tgbonum} %cambiare font all'intero documento




\title{{\color{red} doc11}}
\author{Antonio Maulucci\\ \href{http://www.antomau.com}{antomau.com}}
\date{February 2017}

\begin{document}

\maketitle

\section{Testo colorato}

\LaTeX\ permette di colorare il testo

\subsection{Importare il package}

Per poter scrivere a colori occorre importare il package color

\begin{verbatim}
    \usepackage{color}
\end{verbatim}

\subsection{Testo Rosso}

Per scrivere un testo in colore
{\color{red}rosso}
occorre una sintassi particolare che è:

\begin{verbatim}
    { \color{NomeColore} testoColorato}
\end{verbatim}

in questo caso particolare abbiamo usato

\begin{verbatim}
    {\color{red}rosso}
\end{verbatim}



\subsection{Testo blu}

{\color{blue}Questo testo è scritto in blu}

\subsection{Testo verde}

{\color{green}Questo testo è scritto in verde}

\subsection{Quali colori posso usare?}

Per comprendere quali colori è possibile usare consulare la pagina:
\url{https://en.wikibooks.org/wiki/LaTeX/Colors}



\vspace{2cm}

\section{Link colorati}

\LaTeX\ permette di colorare i Link

\subsection{Importare il package per i link}

Come abbiamo visto nella sezione precedente per poter utilizzare i link in \LaTeX\ dobbiamo importare il relativo package ovvero \textbf{hyperref}

\subsection{Utilizzo di hyperref}

Dopo aver importato il suddetto package dobbiamo impostare il colore dei link in questo modo:

\begin{verbatim}
    \hypersetup{colorlinks=true, linkcolor=blue, urlcolor=blue}
\end{verbatim}

ovviamente dobbiamo aver importato il package \textit{color}.


\vspace{1cm}


\section{Usare un font diverso}

{\fontfamily{qbk}\selectfont
This text uses a different font typeface
}
\\\\
{\fontfamily{lmtt}\selectfont
This text uses a different font typeface
}
\\\\
{\fontfamily{qag}\selectfont
This text uses a different font typeface
}
\\\\
This text uses the basic font


\subsection{Quali font posso usare?}

Per vedere quali font si possono usare nell'editor \href{http://www.it.sharelatex.com}{Sharelatex} consultare

{\fontfamily{qag}\selectfont
\href{https://it.sharelatex.com/learn/Font_typefaces}{questa pagina}
}



\subsection{Come si usano?}

Si usano in questo modo:
\begin{verbatim}
    {\fontfamily{NOME_FAMILY}\selectfont
    TESTO FORMATTATO DIVERSAMENTE
    }
\end{verbatim}


\subsection{Posso usare un
{\fontfamily{qcs}\selectfont
font
} %end of font
esterno?
}

Non con \LaTeX\ ma con X\LaTeX\ .



\vspace{1cm}

\section{Impostare un font generale per la pgina}
Occorre utilizzare il seguente comando:
\begin{verbatim}
    \usepackage{Font_package_name}
\end{verbatim}

ad esempio questa pagina contiene la dichiarazione:

\begin{verbatim}
    \usepackage{tgbonum}
\end{verbatim}


\vspace{4cm}
\begin{center}
{\fontfamily{qag}\selectfont
\copyright \sffamily{Antonio Maulucci 2017}
}
\end{center}
\end{document}
