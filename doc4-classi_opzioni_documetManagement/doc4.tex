\documentclass[a4paper, 12pt]{article}
\usepackage[utf8]{inputenc}
\usepackage[T1]{fontenc}
\usepackage[italian]{babel}

\title{doc4}
\author{Antonio Maulucci \and Antonio Maulucci}
\date{\dots February 2017}


\begin{document}

\maketitle

\backslash documentclass $[$ opzioni $] \{$ classe $\}$

\section{Le classi di documento}


ARTICLE: per scrivere articoli

BOOK: per scrivere libri

LETTER: per scrivere lettere

ci sono altre classi extra

\section{Le opzioni di documento}

10pt, 11pt, 12pt = impostano la dimensione del font principale

a4paper, a5paper... = definiscono la dimensione del foglio (la dimensione predefinita è letterpaper)

onside, twoside = documento a singola o doppia facciata

PER ALTRE VOCI CONSULTARE IL MANUALE A PAGINA 26

Si possono usare più opzioni contemporaneamente seprando i var valori mediante la ','

\section{La gestione del documento}

\subsection{I margini}

Per la gestione dei margini si deve usare il package "geometry"

Creiamo un documento che ha

formato a4

marigini superiori e inferiori di 3cm

sinistro e destro di 3.5 cm

destinare alla rilegatura 5 mm

\backslash geometry $\{$ a4paper,top=3cm,bottom=3cm,left=3.5cm,right=3.5cm,heightrounded,bindingoffset=5mm $\}$

\subsection{Interlinea}

\backslash singlespacing = 1 scartamento

\backslash onehalfspacing = 1.5 scartamento

\backslash doublespacing = 2 scartamento

\backslash linespread $\{$ x $\}$ = x * fattore di scala del documento

\end{document}
