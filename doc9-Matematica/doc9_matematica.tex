\documentclass[a4paper]{article}
\usepackage[T1]{fontenc}
\usepackage[utf8]{inputenc}
\usepackage[italian]{babel}
\usepackage{geometry}
\geometry{a4paper,top=0.5cm,bottom=2cm,left=2cm,right=2cm,bindingoffset=5mm}

\title{doc9 - La Matamatica con \LaTeX\ }
\author{Antonio Maulucci}
\date{February 2017}

\begin{document}

\maketitle

\section{Una formula matematica}
\subsection{Una semplice formula}

\( x+y+3=17 \)

\subsection{La stessa formula scritta matematicamente}

$ x + y +3 = 17 $ % gli spazi non influiscono sull'output finale

\subsection{Esito}

Non cambia nulla ma la seconda opzione è quella preferibile!

\subsection{Ambiente displayMath}


Il seguente ambiente dispone la formula al centro della pagina permettendo di inserire questa indipendentemente dai caratteri \$:

\begin{displaymath}
x + y + 3 = 17
% lasciando una linea bianca all'interno di questo ambiente il compilatore restituirà un errore
\end{displaymath}

\subsection{Ambiente math semplificato}

Questo è un metodo per scrivere una formula semplificando l'utilizzo dell'ambiente math:

\[ x+ y +3 = 17 \]

Un'ulteriore semplificazione è:

$$x + y +3 = 17$$




\section{Indici ed esponenti}

\subsection{Esponenti}

Scrivere x al quadrato:

$$ x^2 $$

Esempio di formula:

$$ (x^2 + x^3)^4 = 12 $$

\subsection{Indici}

$$ x_2 $$
$$ x_3 $$

\section{Testo in una formula}
$$ f(x)> 0 \mbox{ è vero se } x=3 $$ % N.B.: lasciare uno spazio all'inizio e alla fine nelle parentesi graffe

Se non si lascia lo spazio all'interno delle parentesi graffe accade:

$$ f(x)>0 \mbox{è vero se} x=3$$

\section{Lettere e simboli matematici}

Consulare la guida \LaTeX (imparare latex) a pagina 39

Alcuni esempi:

$ \alpha\ \beta\ \gamma\ \delta\ \eta\ \theta\ \epsilon\\ \Pi\ \Theta\ \Delta\ $

La costante matematica $\Pi$ vale $3,14...$

\section{Operatori}

$ \pm\ \mp\ \times\ \div\ \ast\ \star\ \bullet\ \cup\ \not\ \surd\ \nabla $

guida pag. 40

\section{Accenti}

$
\hat{a}\\
\bar{x}\\
\vec{y}\\
\dot{a}\\
\ddot{x}\\
\overrightarrow{AB}
$

guida pag. 42/43

\section{Frazioni}

$
\frac{(x^2+x^3)^4}{(x^6-x^3)^5}
$

\section{Radici}

$
\sqrt[9]{(x^6)^(x+3)}
$

\section{Delimitatori}

\subsection{Matrici}

$
\left[
\begin{array}{cc}
    a_{11} & a_{12} \\ a_{23} & a_{24}
\end{array}
\right]
$

\subsection{Matrici con numeri}

$
\left[
\begin{array}{ccc} %cc come per le tabelle indica l'allineamento di ogni colonna
    12 & 27 & 24 \\ 22 & 34 & 78
\end{array}
\right]
$

\subsection{Sistema di equazioni}

$
|x| =
\left\{
\begin{array}{r}
    x^2 = 3x + 2 \\
    x^3 = 2x +4
\end{array}
\right.
$

\section{Simboli a grandezza variabile}

\subsection{Sommatoria}

$
\sum_{i=0}^{+\infty} x_i
$

\subsection{Integrale}

$
\int x^2 dx = \frac{x^3}{3} + c
$

\subsubsection{Integrale definito}
$
\int_a^b x^2 dx
$

\subsubsection{Integrale circolare}
$
\oint x^2
$

\subsubsection{Integrale triplo}
$ \int \!\!\!\! \int \!\!\!\! \int x^2 dx $


\section{Limitii}

$
\lim_{x \rightarrow +\infty} (x^2+3x)
$

\section{Limite di logaritmo}

$$
\lim_{x \rightarrow +\infty} \ln x^2 = +\infty
$$

\section{Le tabelle e l'ambiente Array}



\vspace{6cm}
\begin{center}
\copyright Antonio Maulucci 2017
\end{center}
\end{document}
